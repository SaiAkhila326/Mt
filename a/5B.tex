%iffalse
\let\negmedspace\undefined
\let\negthickspace\undefined
\documentclass[journal,12pt,twocolumn]{IEEEtran}
\usepackage{cite}
\usepackage{amsmath,amssymb,amsfonts,amsthm}
\usepackage{algorithmic}
\usepackage{graphicx}
\usepackage{textcomp}
\usepackage{xcolor}
\usepackage{txfonts}
\usepackage{listings}
\usepackage{enumitem}
\usepackage{mathtools}
\usepackage{gensymb}
\usepackage{comment}
\usepackage[breaklinks=true]{hyperref}
\usepackage{tkz-euclide} 
\usepackage{listings}
\usepackage{gvv}                                        
%\def\inputGnumericTable{}                                 
\usepackage[latin1]{inputenc}                                
\usepackage{color}                                            
\usepackage{array}                                            
\usepackage{longtable}                                       
\usepackage{calc}                                             
\usepackage{multirow}                                         
\usepackage{hhline}                                           
\usepackage{ifthen}                                           
\usepackage{lscape}
\usepackage{tabularx}
\usepackage{array}
\usepackage{float}


\newtheorem{theorem}{Theorem}[section]
\newtheorem{problem}{Problem}
\newtheorem{proposition}{Proposition}[section]
\newtheorem{lemma}{Lemma}[section]
\newtheorem{corollary}[theorem]{Corollary}
\newtheorem{example}{Example}[section]
\newtheorem{definition}[problem]{Definition}
\newcommand{\BEQA}{\begin{eqnarray}}
\newcommand{\EEQA}{\end{eqnarray}}
\newcommand{\define}{\stackrel{\triangle}{=}}
\theoremstyle{remark}
\newtheorem{rem}{Remark}

% Marks the beginning of the document
\begin{document}
\bibliographystyle{IEEEtran}
\vspace{3cm}

\title{5B}
\author{ee24btech11055 - Sai Akhila Reddy Turpu}
\maketitle
\newpage
\bigskip

\renewcommand{\thefigure}{\theenumi}
\renewcommand{\thetable}{\theenumi}
1. The coefficients of $x^p$ and $x^q$ in the expansion of $(1+x)^{p+q}$ are:
\hfill{(2002)}
\begin{enumerate}[label=(\alph*)]
\item equal
\item equal with opposite signs
\item reciprocals of each other
\item none of these
\end{enumerate}
2. If the sum of coefficients in the expansion of $(a+b)^n$ is 4096, then the greatest coefficient in the expansion is:  
\hfill{(2002)}
\begin{enumerate}[label=(\alph*)]
\item $1594$
\item $792$
\item $924$
\item $2924$
\end{enumerate}
3. The positive integer just greater than \\
$(1+0.0001)^{10000}$ is: 
\hfill{(2002)}
\begin{enumerate}[label=(\alph*)]
\item $4$
\item $5$
\item $2$
\item $3$
\end{enumerate}
4. r and n are positive integers, $r>1, n>2$ and coefficient of $(r+2)^{th}$ term and $(3r)^{th}$ term in the expansion of $(1+x)^{2n}$ are equal, then n equals:

\hfill{(2002)}
\begin{enumerate}[label=(\alph*)]
\item $3r$
\item $3r+1$
\item $2r$
\item $2r+1$
\end{enumerate}
5. If $a_n = \sqrt{7+\sqrt{7+\sqrt{7+...}}}$ having n radical signs, then by methods of mathematical induction, which is true?
\hfill{(2002)}
\begin{enumerate}[label=(\alph*)]
\item $a_n > 7$  $\forall$ $n \ge 1$
\item $a_n < 7$  $\forall$ $n \ge 1$
\item $a_n < 4$  $\forall$ $n \ge 1$
\item $a_n < 3$  $\forall$ $n \ge 1$
\end{enumerate}
6. If x is positive, the first negative term in the expansion of $(1+x)^{27/5}$ is:
\hfill{(2003)}
\begin{enumerate}[label=(\alph*)]
\item 6th term
\item 7th term
\item 5th term
\item 8th term
\end{enumerate}
7. The number of integral terms in the expansion of $(\sqrt{3}+\sqrt[8]{5})^{256}$ is:
\hfill{(2003)}
\begin{enumerate}[label=(\alph*)]
\item $35$
\item $32$
\item $33$
\item $34$
\end{enumerate}
\end{document}
