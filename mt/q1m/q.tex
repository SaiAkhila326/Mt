\let\negmedspace\undefined
\let\negthickspace\undefined
\documentclass[journal]{IEEEtran}
\usepackage[a5paper, margin=10mm, onecolumn]{geometry}
%\usepackage{lmodern} % Ensure lmodern is loaded for pdflatex
\usepackage{tfrupee} % Include tfrupee package

\setlength{\headheight}{1cm} % Set the height of the header box
\setlength{\headsep}{0mm}     % Set the distance between the header box and the top of the text

\usepackage{gvv-book}
\usepackage{gvv}
\usepackage{cite}
\usepackage{amsmath,amssymb,amsfonts,amsthm}
\usepackage{algorithmic}
\usepackage{graphicx}
\usepackage{textcomp}
\usepackage{xcolor}
\usepackage{txfonts}
\usepackage{listings}
\usepackage{enumitem}
\usepackage{mathtools}
\usepackage{gensymb}
\usepackage{comment}
\usepackage[breaklinks=true]{hyperref}
\usepackage{tkz-euclide} 
\usepackage{listings}
% \usepackage{gvv}                                        
\def\inputGnumericTable{}                                 
\usepackage[latin1]{inputenc}                                
\usepackage{color}                                            
\usepackage{array}                                            
\usepackage{longtable}                                       
\usepackage{calc}                                             
\usepackage{multirow}                                         
\usepackage{hhline}                                           
\usepackage{ifthen}                                           
\usepackage{lscape}

\setlength{\parindent}{0pt}

\begin{document}
\bibliographystyle{IEEEtran}
\title{1.4.12}
\author{EE24BTECH11055 - Sai Akhila Reddy Turpu}
{\let\newpage\relax\maketitle}
\renewcommand{\thefigure}{\theenumi}
\renewcommand{\thetable}{\theenumi}
\setlength{\intextsep}{10pt} % Space between text and floats
\numberwithin{equation}{enumi}
\numberwithin{figure}{enumi}
\renewcommand{\thetable}{\theenumi}
Question:\\
The position vector of the point which divides the join of points $2a-3b$ and $a + b$ in the ratio $3:1$ is:\\
\solution{


Using Section Formula(k=3)\\

\begin{align}
	C &= (kB+A)/k+1 \\
	C &= \frac{1}{3+1} \brak{3B + A} \\
\implies	C &= \frac{1}{4}\brak{ \brak{3a + 3b} + \brak{2a - 3b}} \\
	C &= \frac{5}{4} a
\end{align}
}
\begin{table}[h]
    \centering
    \begin{tabular}{|m{5em}|m{10em}|}
    \hline
    \textbf{Vector} &\textbf{Coordinates} \\
    \hline
         $\vec{A}$ & $2\vec{a}-3\vec{b}$ \\
    \hline
        $\vec{B}$ & $\vec{a}+\vec{b}$ \\
    \hline
	$\vec{C}$ & $\frac{5}{4}$ $\vec{a}$ \\
    \hline
\end{tabular}

    \caption{Given Values}
    \label{tab:1}
\end{table}
\begin{table}[h]
	\centering
	\begin{tabular}{|c|c|c|}
\hline
Original $a$ & Vector obtained after applying section formula $\frac{5}{4}a$ & Verification \\ 
\hline
1.0000 & 1.2500 & True \\ 
2.0000 & 2.5000 & True \\ 
\hline
\end{tabular}

	\caption{Verified values}
	\label{tab:2}
\end{table}

\end{document}
