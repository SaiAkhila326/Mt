\documentclass{beamer}
\mode<presentation>
\usepackage{amsmath}
\usepackage{amssymb}
\usepackage{adjustbox}
\usepackage{subcaption}
\usepackage{enumitem}
\usepackage{multicol}
\usepackage{mathtools}
\usepackage{listings}
\usepackage{url}
\usepackage{array}
\def\UrlBreaks{\do\/\do-}
\usetheme{Boadilla}
\usecolortheme{lily}
\setbeamertemplate{footline}{
  \leavevmode%
  \hbox{%
  \begin{beamercolorbox}[wd=\paperwidth,ht=2.25ex,dp=1ex,right]{author in head/foot}%
    \insertframenumber{} / \inserttotalframenumber\hspace*{2ex} 
  \end{beamercolorbox}}%
  \vskip0pt%
}
\setbeamertemplate{navigation symbols}{}

\providecommand{\nCr}[2]{\,^{#1}C_{#2}} % nCr
\providecommand{\nPr}[2]{\,^{#1}P_{#2}} % nPr
\providecommand{\mbf}{\mathbf}
\providecommand{\pr}[1]{\ensuremath{\Pr\left(#1\right)}}
\providecommand{\qfunc}[1]{\ensuremath{Q\left(#1\right)}}
\providecommand{\sbrak}[1]{\ensuremath{{}\left[#1\right]}}
\providecommand{\lsbrak}[1]{\ensuremath{{}\left[#1\right.}}
\providecommand{\rsbrak}[1]{\ensuremath{{}\left.#1\right]}}
\providecommand{\brak}[1]{\ensuremath{\left(#1\right)}}
\providecommand{\lbrak}[1]{\ensuremath{\left(#1\right.}}
\providecommand{\rbrak}[1]{\ensuremath{\left.#1\right)}}
\providecommand{\cbrak}[1]{\ensuremath{\left\{#1\right\}}}
\providecommand{\lcbrak}[1]{\ensuremath{\left\{#1\right.}}
\providecommand{\rcbrak}[1]{\ensuremath{\left.#1\right\}}}
\theoremstyle{remark}
\newtheorem{rem}{Remark}
\newcommand{\sgn}{\mathop{\mathrm{sgn}}}
\providecommand{\abs}[1]{\left\vert#1\right\vert}
\providecommand{\res}[1]{\Res\displaylimits_{#1}} 
\providecommand{\norm}[1]{\lVert#1\rVert}
\providecommand{\mtx}[1]{\mathbf{#1}}
\providecommand{\mean}[1]{E\left[ #1 \right]}
\providecommand{\fourier}{\overset{\mathcal{F}}{ \rightleftharpoons}}
\providecommand{\system}{\overset{\mathcal{H}}{ \longleftrightarrow}}
%\renewcommand{\solution}[1]{\textbf{Solution:} #1} % Correct definition of solution command
\renewcommand{\solution}[1]{\textbf{Solution:} #1} % Redefine solution command
\providecommand{\dec}[2]{\ensuremath{\overset{#1}{\underset{#2}{\gtrless}}}}
\newcommand{\myvec}[1]{\ensuremath{\begin{pmatrix}#1\end{pmatrix}}}
\let\vec\mathbf

\lstset{
frame=single, 
breaklines=true,
columns=fullflexible
}

\numberwithin{equation}{section}

\title{Presentation Template}
\author{Sai Akhila\\ Dept. of Electrical Engg.,\\IIT Hyderabad.}
\date{\today} 

\begin{document}

\begin{frame}
\titlepage
\end{frame}

\section*{Outline}
\begin{frame}
\tableofcontents
\end{frame}

\section{Problem}
\begin{frame}
\frametitle{Problem Statement}
The position vector of the point which divides the join of points \(2a-3b\) and \(a + b\) in the ratio \(3:1\) is:
\end{frame}

\section{Solution}
\subsection{Given values}
\begin{frame}
\frametitle{Given values}
\begin{table}[h]
    \centering
    \begin{tabular}{|m{5em}|m{10em}|}
    \hline
    Vector &Coordinates \\
    \hline
         $A$ & $2a-3b$ \\
    \hline
        $B$ & $a+b$ \\
    \hline
	$C$ & $\frac{5}{4}$ $a$ \\
    \hline
\end{tabular}

    \caption{Given Values}
    \label{tab:1}
\end{table}
\end{frame}

\subsection{Section formula}
\begin{frame}
\frametitle{Section formula}
\solution{
Using Section Formula (\(k=3\)):
\begin{align}
	C &= \frac{kB+A}{k+1} \\
	C &= \frac{1}{3+1} \brak{3B + A} \\
    \implies	C &= \frac{1}{4}\brak{ \brak{3a + 3b} + \brak{2a - 3b}} \\
	C &= \frac{5}{4} a
\end{align}
}
\end{frame}

\begin{frame}
    The code in 
    {\footnotesize
    \url{https://github.com/SaiAkhila326/Mt/blob/master/mt/q1m/codes/verify.py}
    }
    verifies the equation.
\end{frame}


\subsection{Verified values}
\begin{frame}
\frametitle{Verified values}
\begin{table}[h]
	\centering
	


\begin{tabular}{|c|c|c|c|c|c|}
\hline
Row & a & b & A = 2a - 3b & B = a + b & Resultant Vector \\
\hline
Row 1 & 1.00 & 3.00 & -7.00 & 4.00 & 1.25 \\
Row 2 & 2.00 & 4.00 & -8.00 & 6.00 & 2.50 \\

\hline
\end{tabular}




	\caption{Verified values}
	\label{tab:2}
\end{table}
\end{frame}
\end{document}

