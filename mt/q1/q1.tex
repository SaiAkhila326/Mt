\let\negmedspace\undefined
\let\negthickspace\undefined
\documentclass[journal]{IEEEtran}
\usepackage[a5paper, margin=10mm, onecolumn]{geometry}
%\usepackage{lmodern} % Ensure lmodern is loaded for pdflatex
\usepackage{tfrupee} % Include tfrupee package

\setlength{\headheight}{1cm} % Set the height of the header box
\setlength{\headsep}{0mm}     % Set the distance between the header box and the top of the text

\usepackage{gvv-book}
\usepackage{gvv}
\usepackage{cite}
\usepackage{amsmath,amssymb,amsfonts,amsthm}
\usepackage{algorithmic}
\usepackage{graphicx}
\usepackage{textcomp}
\usepackage{xcolor}
\usepackage{txfonts}
\usepackage{listings}
\usepackage{enumitem}
\usepackage{mathtools}
\usepackage{gensymb}
\usepackage{comment}
\usepackage[breaklinks=true]{hyperref}
\usepackage{tkz-euclide} 
\usepackage{listings}
% \usepackage{gvv}                                        
\def\inputGnumericTable{}                                 
\usepackage[latin1]{inputenc}                                
\usepackage{color}                                            
\usepackage{array}                                            
\usepackage{longtable}                                       
\usepackage{calc}                                             
\usepackage{multirow}                                         
\usepackage{hhline}                                           
\usepackage{ifthen}                                           
\usepackage{lscape}
\begin{document}
\bibliographystyle{IEEEtran}
\title{1.4.12}
\author{EE24BTECH11055 - Sai Akhila Reddy Turpu}
{\let\newpage\relax\maketitle}
\renewcommand{\thefigure}{\theenumi}
\renewcommand{\thetable}{\theenumi}
\setlength{\intextsep}{10pt} % Space between text and floats
\numberwithin{equation}{enumi}
\numberwithin{figure}{enumi}
\renewcommand{\thetable}{\theenumi}
\textbf{Question}:\\
The position vector of the point which divides the join of points $2\vec{a}-3\vec{b}$ and $\vec{a} + \vec{b}$ in the ratio $3:1$ is:
\textbf{Solution:}
\begin{table}[h]
    \centering
    \begin{tabular}{|m{5em}|m{10em}|}
    \hline
    Vector &Coordinates \\
    \hline
         $A$ & $2a-3b$ \\
    \hline
        $B$ & $a+b$ \\
    \hline
	$C$ & $\frac{5}{4}$ $a$ \\
    \hline
\end{tabular}

    \caption{Given Vectors}
    \label{tab:1}
\end{table}
Given\\
Let $\vec{C}$ divides $\vec{A}$ and $\vec{B}$ in the ratio 3:1\\
Using Section Formula(k=3)
\begin{align}
\vec{C} = \frac{1}{3+1} \brak{3\myvec{B} + \myvec{A}} \\
\vec{C} = \brak{\frac{1}{4}} \brak{3\myvec{a} + \myvec{b}} + \brak{2\myvec{a} - 3\myvec{b}} \\
\vec{C} = \frac{1}{4} \brak{5\myvec{a}}

\end{align}
Thus coordinates of $\vec{C}$ is 5/4\vec{a}. 

\end{document}
