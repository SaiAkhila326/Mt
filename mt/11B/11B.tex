%iffalse
\let\negmedspace\undefined
\let\negthickspace\undefined
\documentclass[journal,12pt,twocolumn]{IEEEtran}
\usepackage{cite}
\usepackage{amsmath,amssymb,amsfonts,amsthm}
\usepackage{algorithmic}
\usepackage{graphicx}
\usepackage{textcomp}
\usepackage{xcolor}
\usepackage{txfonts}
\usepackage{listings}
\usepackage{enumitem}
\usepackage{mathtools}
\usepackage{gensymb}
\usepackage{comment}
\usepackage[breaklinks=true]{hyperref}
\usepackage{tkz-euclide} 
\usepackage{listings}
\usepackage{gvv}                                        
%\def\inputGnumericTable{}                                 
\usepackage[latin1]{inputenc}                                
\usepackage{color}                                            
\usepackage{array}                                            
\usepackage{longtable}                                       
\usepackage{calc}                                             
\usepackage{multirow}                                         
\usepackage{hhline}                                           
\usepackage{ifthen}                                           
\usepackage{lscape}
\usepackage{tabularx}
\usepackage{array}
\usepackage{float}
\usepackage{multicol}
\usepackage{amsmath}


\newtheorem{theorem}{Theorem}[section]
\newtheorem{problem}{Problem}
\newtheorem{proposition}{Proposition}[section]
\newtheorem{lemma}{Lemma}[section]
\newtheorem{corollary}[theorem]{Corollary}
\newtheorem{example}{Example}[section]
\newtheorem{definition}[problem]{Definition}
\newcommand{\BEQA}{\begin{eqnarray}}
\newcommand{\EEQA}{\end{eqnarray}}
\newcommand{\define}{\stackrel{\triangle}{=}}
\theoremstyle{remark}
\newtheorem{rem}{Remark}

% Marks the beginning of the document
\begin{document}
\bibliographystyle{IEEEtran}
\vspace{3cm}

\title{11B(36-37)}
\author{ee24btech11055 - Sai Akhila Reddy Turpu}
\maketitle
\newpage
\bigskip

\renewcommand{\thefigure}{\theenumi}
\renewcommand{\thetable}{\theenumi}
\begin{enumerate}[start=36]
		


	\item For $x\in\mathbb{R}$, $f(x)=\abs{\log 2-\sin(x)}$ and 
	$g(x)=f(f(x))$, then 
	\hfill{(JEE M 2016)}
	\begin{enumerate}[label=(\alph*)]
		\item $g'(0) = -\cos(\log 2)$
		\item g is differentiable at $x=0$ and $g'(0) = -\sin(\log 2)$
		\item g is not differentiable at $x=0$
		\item $g'(0) =  \cos(\log 2)$
	\end{enumerate}

\item $\lim_{x\to\infty} \brak{\frac{(n+1)(n+2)...3n}{n^{2n}}}^{\frac{1}{n}} $ is equal to:
	\hfill{(JEE M 2016)}
	\begin{enumerate}[label=(\alph*)]
			\begin{multicols}{2}
		\item $\frac{9}{e^2}$
		\item $3\log 3 - 2$
		\item $\frac{18}{e^4}$
		\item $\frac{27}{e^2}$
			\end{multicols}
	\end{enumerate}
\item Let p = $\lim_{x\to0^+} \brak{1+\tan^2(\sqrt{x})}^{\frac{1}{2x}} $ then $\log p$ is equal to:
	\hfill{(JEE M 2016)}
	\begin{enumerate}[label=(\alph*)]
			\begin{multicols}{2}
		\item $\frac{1}{2}$
		\item $\frac{1}{4}$
		\item $2$
		\item $1$
			\end{multicols}
	\end{enumerate}
\item $\lim_{x\to\frac{\pi}{2}} \frac{\cot(x)-\cos(x)}{(\pi -2x)^3} $ equals 
	\hfill{(JEE M 2017)}
	\begin{enumerate}[label=(\alph*)]
			\begin{multicols}{4}
		\item $\frac{1}{4}$
		\item $\frac{1}{24}$
		\item $\frac{1}{16}$
		\item $\frac{1}{8}$
			\end{multicols}
	\end{enumerate}
\item For each $t\in\mathbb{R}$, let $\sbrak{t}$ be the greatest integer less than or equal to $t$. Then 
	$\lim_{x\to0^+} x\brak{ \sbrak{\frac{1}{x}}+\sbrak{\frac{2}{x}}+...+\sbrak{ \frac{15}{x}}} $
	\hfill{(JEE M 2018)}
	\begin{enumerate}[label=(\alph*)]
			\begin{multicols}{2}
		\item is equal to 15
		\item is equal to 120
		\item does not exist$(in \mathbb{R})$
		\item is equal to 0
			\end{multicols}
	\end{enumerate}
\item For S =  $t\in\mathbb{R}:f(x)=\abs{x-\pi}(e^{\abs{x}}-1)\sin(\abs{x})$ is not differentiable at t. Then the set S is equal to:

	\hfill{(JEE M 2018)}
	\begin{enumerate}[label=(\alph*)]
			\begin{multicols}{2}
		\item {$0$}
		\item ${\pi}$
		\item ${0,\pi}$
		\item ${\emptyset}$(an empty set)
			\end{multicols}
	\end{enumerate}
	\item  
	$\lim_{y\to0} \frac{\sqrt{1+\sqrt{1+y^4}}-\sqrt{2}}{y^4} $
	\hfill{(JEE M 2019- 9 Jan(M))}
	\begin{enumerate}[label=(\alph*)]
		\item exists and equals $\frac{1}{4\sqrt{2}}$
		\item exists and equals $\frac{1}{2\sqrt{2}(\sqrt{2}+1)}$
		\item exists and equals $\frac{1}{2\sqrt{2}}$
		\item does not exist
	\end{enumerate}
\item Let $f : \mathbb{R}\to\mathbb{R}$ is a function defined as

	\hfill{(JEE M 2019- 9 Jan(M))}

	\begin{equation}
	    f(x)= 
	    \begin{cases}
	    5, \hfill{if x \le 1}\\
	    a + bx, \hfill{if 1 < 3}\\
	    b + 5x, \hfill{if 3 \le x < 5}\\
	    30, \hfill{if x \ge 5}
	    \end{cases}
	\end{equation}

	\begin{enumerate}[label=(\alph*)]
		\item continuous if $a=5$ and $b=5$
		\item continuous if $a=-5$ and $b=10$
		\item continuous if $a=0$ and $b=5$
		\item not continuous for any values of a and b
	\end{enumerate}
\item If the function f defined on $ \brak{ \frac{\pi}{3}, \frac{\pi}{6}}$ by
	\begin{equation}
	    f(x)= 
	    \begin{cases}
		    \frac{\sqrt{2} cos(x)-1}{cot(x)-1}, \hfill{x \ne \frac{\pi}{4}}\\
		    k, \hfill{x = \frac{\pi}{4}}\\
	    \end{cases}
	\end{equation}
	\hfill{(JEE M 2019- 9 April(M))}
	\begin{enumerate}[label=(\alph*)]
			\begin{multicols}{4}
		\item $2$
		\item $\frac{1}{2}$
		\item $1$
		\item $\frac{1}{\sqrt{2}}$
			\end{multicols}
	\end{enumerate}
\item Let $f(x)= 15-\abs{x-10}, x\to\mathbb{R}$. Then the set of all values of x, at which the function, g(x)=f(f(x)) is not differentiable, is:

	\hfill{(JEE M 2019- 9 April(M))}
	\begin{enumerate}[label=(\alph*)]
			\begin{multicols}{2}
		\item $\cbrak{5,10,15}$
		\item $\cbrak{10,15}$
		\item $\cbrak{5,10,15,20}$
		\item $\cbrak{10}$
			\end{multicols}
	\end{enumerate}
\end{enumerate}
\end{document}

