\let\negmedspace\undefined
\let\negthickspace\undefined
\documentclass[journal]{IEEEtran}
\usepackage[a5paper, margin=10mm, onecolumn]{geometry}
%\usepackage{lmodern} % Ensure lmodern is loaded for pdflatex
\usepackage{tfrupee} % Include tfrupee package

\setlength{\headheight}{1cm} % Set the height of the header box
\setlength{\headsep}{0mm}     % Set the distance between the header box and the top of the text

\usepackage{gvv-book}
\usepackage{gvv}
\usepackage{cite}
\usepackage{amsmath,amssymb,amsfonts,amsthm}
\usepackage{algorithmic}
\usepackage{graphicx}
\usepackage{textcomp}
\usepackage{xcolor}
\usepackage{txfonts}
\usepackage{listings}
\usepackage{enumitem}
\usepackage{mathtools}
\usepackage{gensymb}
\usepackage{comment}
\usepackage[breaklinks=true]{hyperref}
\usepackage{tkz-euclide} 
\usepackage{listings}
% \usepackage{gvv}                                        
\def\inputGnumericTable{}                                 
\usepackage[latin1]{inputenc}                                
\usepackage{color}                                            
\usepackage{array}                                            
\usepackage{longtable}                                       
\usepackage{calc}                                             
\usepackage{multirow}                                         
\usepackage{hhline}                                           
\usepackage{ifthen}                                           
\usepackage{lscape}
\begin{document}

\bibliographystyle{IEEEtran}
\vspace{3cm}

\title{08-27-2021-shift-2(1-15)}
\author{EE24BTECH11055 - Sai Akhila Reddy Turpu}
% \maketitle
% \newpage
% \bigskip
{\let\newpage\relax\maketitle}



\renewcommand{\thefigure}{\theenumi}
\renewcommand{\thetable}{\theenumi}
\setlength{\intextsep}{10pt} % Space between text and floats


\numberwithin{equation}{enumi}
\numberwithin{figure}{enumi}
\renewcommand{\thetable}{\theenumi}

\begin{enumerate}
		


	\item The angle between the straight lines, whose direction cosines are given by the equations $2l + 2m - n = 0$ and $mn + nl + lm =0$, is : 
	\begin{enumerate}
	\begin{multicols}{2}
		\item $\frac{\pi}{2}$ 
		\item $\pi - \cos^{-1}\brak{\frac{4}{9}} $
		\item $\cos^{-1}\brak{\frac{8}{9}} $
		\item $\frac{\pi}{3}$
		\end{multicols}
	\end{enumerate}

\item Let $A = \myvec{
		\sbrak{x + 1} & \sbrak{x + 2} & \sbrak{x + 3} \\
		\sbrak{x} & \sbrak{x + 3} & \sbrak{x + 3} \\
		\sbrak{x} & \sbrak{x + 2} & \sbrak{x + 4}
		}
		$ where $\sbrak{t}$ denotes the greatest integer less than or equal to $t$. If $det(A) = 192$,  then the set of values of $x$ is the interval:  

	
	\begin{enumerate}
	\begin{multicols}{2}
		\item $[68, 69)$
		\item $[62, 63)$
		\item $[65, 66)$
		\item $[60, 61)$
	\end{multicols}
	\end{enumerate}
\item Let $M$ and $m$ respectively be the maximum and minimum values of the function $f\brak{x} = \tan^{-1}\brak{\sin x+\cos x}$ in $\sbrak{0,\frac{\pi}{2}}$, then the value of $\tan(M-m)$ is equal to:
	
	\begin{enumerate}
		\begin{multicols}{2}
		\item $2 + \sqrt{3}$
		\item $2 - \sqrt{3}$
		\item $3 + 2\sqrt{2}$
		\item $3 - 2\sqrt{2}$
			\end{multicols}
	\end{enumerate}
\item Each of the persons $A$ and $B$ independently tosses three fair coins. The probability that both of them get the same number of heads is: 
        
	\begin{enumerate}
			\begin{multicols}{4}
		\item $\frac{1}{8}$
		\item $\frac{5}{8}$
		\item $\frac{5}{16}$
		\item $1$
			\end{multicols}
	\end{enumerate}
\item A differential equation representing the family of parabolas with axis parallel to y-axis and whose length of latus rectum is the distance of the point $\brak{2,-3}$ from the line $3x + 4y = 5$, is given by:
	
	\begin{enumerate}
			\begin{multicols}{2}
			\item $10\frac{d^2y}{dx^2} = 11$
			\item $11\frac{d^2x}{dy^2} = 10$
			\item $10\frac{d^2x}{dy^2} = 11$
			\item $11\frac{d^2y}{dx^2} = 10$
			\end{multicols}
	\end{enumerate}
\item If two tangents drawn from a point $\vec{P}$ to the parabola $y^2 = 16\brak{x-3}$ are at right angles, then the locus of point $\vec{P}$ is: 
	
	\begin{enumerate}
		\begin{multicols}{2}
		\item $x + 3 = 0$
		\item $x + 1 = 0$
		\item $x + 2 = 0$
		\item $x + 4 = 0$
		\end{multicols}
	\end{enumerate}
	\item  The equation of the plane passing through the line of intersection of planes $\vec{\overrightarrow{r}} \cdot \brak{\hat{i} + \hat{j} + \hat{k}} = 1$ and $\vec{\overrightarrow{r}} \cdot \brak{2\hat{i} + 3\hat{j} - \hat{k}} + 4 = 0$ and parallel to the x-axis is: 
	
	\begin{enumerate}
	\begin{multicols}{2}
		\item $\overrightarrow{r} \cdot \brak{\hat{j} - 3\hat{k}} + 6 = 0$
		\item $\overrightarrow{r} \cdot \brak{\hat{i} + 3\hat{k}} + 6 = 0$
		\item $\overrightarrow{r} \cdot \brak{\hat{i} - 3\hat{k}} + 6 = 0$
		\item $\overrightarrow{r} \cdot \brak{\hat{j} - 3\hat{k}} - 6 = 0$
	\end{multicols}
	\end{enumerate}
\item If the solution curve of the differential equation $\brak{2x - 10y^3}dy + ydx = 0$, passes through the points $  \brak{0,1}$ and $\brak{2,\beta}$, then $\beta$ is a root of the equation: 

	

	\begin{enumerate}
	\begin{multicols}{2}
		\item $y^5 - 2y - 2 = 0$
		\item $2y^5 - 2y - 1 = 0 $
		\item $2y^5 - y^2 - 2 = 0$
		\item $y^5 -y^2 -1 = 0$
	\end{multicols}
	\end{enumerate}
\item Let $\vec{A}\brak{a, 0}$, $\vec{B}\brak{b, 2b + 1}$ and $\vec{C}\brak{0, b}, b \neq 0, |b| \neq 1$, be points such that the area of the triangle ABC is $1$ sq. unit, then the sum of all possible values of $a$ is: 
	
	\begin{enumerate}
			\begin{multicols}{2}
		\item $\frac{-2b}{b+1}$
		\item $\frac{2b}{b+1}$
		\item $\frac{2b^2}{b+1}$
		\item $\frac{-2b^2}{b+1}$
			\end{multicols}
	\end{enumerate}
\item Let $\sbrak{\lambda}$ be the greatest integer less than or equal to $\lambda$. The set of all values of $\lambda$ for which the system of linear equations $x + y + z = 4$, $3x + 2y + 5z = 3$, $9x + 4y + \brak{28 + \sbrak{\lambda}}z = \sbrak{\lambda}$ has a solution is: 

	
	\begin{enumerate}		
		\item $\mathbb{R}$
		\item $\brak{-\infty,-9} \cup \brak{-9,\infty}$
		\item $[-9,-8)$
		\item $\brak{-\infty,-9} \cup \brak{-8,\infty}$			
	\end{enumerate}
\item The set of all values of $k > -1$, for which the equation $\brak{3x^2 + 4x +3}^2 -\brak{k+1}\brak{3x^2 + 4x +3}\brak{3x^2 + 4x +2} +k\brak{3x^2 + 4x +2}^2 = 0$ has real roots, is:
	
	\begin{enumerate}
			\begin{multicols}{2}
			\item $ \left(1 , \frac{5}{2} \right] $
		\item $ [2, 3) $
		\item $ \left[-\frac{1}{2} , 1 \right) $
		\item $ \left( \frac{1}{2}, \frac{3}{2} \right] - \{1\} $
			\end{multicols}
	\end{enumerate}
\item A box open from top is made from a rectangular sheet of dimension $a \times b$ by cutting squares each of side $x$ from each of the four corners and folding up the flaps. If the volume of the box is maximum, then $x$ is equal to:  

	
	\begin{enumerate}
		\item $\frac{a + b - \sqrt{a^2 + b^2 - ab}}{12}$
		\item $\frac{a + b - \sqrt{a^2 + b^2 + ab}}{6}$
		\item $\frac{a + b - \sqrt{a^2 + b^2 - ab}}{6}$
		\item $\frac{a + b + \sqrt{a^2 + b^2 - ab}}{6}$		
	\end{enumerate}
\item The Boolean expression $\brak{p \land q} \implies \brak{\brak{r \land q} \land p}$ is equivalent to:
	
	\begin{enumerate}
			\begin{multicols}{2}
		\item $\brak{p \land q} \implies \brak{r \land q}$
		\item $\brak{q \land r} \implies \brak{p \land q}$
		\item $\brak{p \land q} \implies \brak{r \lor q}$
		\item $\brak{p \land r} \implies \brak{p \land q}$
			\end{multicols}
	\end{enumerate}
\item Let $\mathbb{Z}$ be the set of all integers,\\
$A = \cbrak{\brak{x, y} \in \mathbb{Z} \times \mathbb{Z} : \brak{x-2}^2 + y^2 \le 4}$,\\
$B = \cbrak{\brak{x, y} \in \mathbb{Z} \times \mathbb{Z} : x^2 + y^2 \le 4}$,\\
$C = \cbrak{\brak{x, y} \in \mathbb{Z} \times \mathbb{Z} : \brak{x-2}^2 + \brak{y-2}^2 \le 4}$,\\
If the total number of relations from $A \cap B$ to $A \cap C$ is $2^p$, then the value of $p$ is: 
	
	\begin{enumerate}
			\begin{multicols}{2}	
		\item $16$
		\item $25$
		\item $49$
		\item $9$
			\end{multicols}
	\end{enumerate}
\item The area of the region bounded by the parabola $\brak{y-2}^2 = \brak{x-1}$, the tangent to it at the point whose ordinate is $3$ and the x-axis is:
 
	\begin{enumerate}
			\begin{multicols}{4}
		\item $9$
		\item $10$
		\item $4$
		\item $6$
			\end{multicols}
	\end{enumerate}
\end{enumerate}
\end{document}
